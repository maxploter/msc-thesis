% !TEX root = thesis.tex

% Valime eesti keele. Infolehel, võõrsõnades ja tsitaatides on ka inglise keel.
\documentclass[english,estonian]{unitartucs/thesis}

\usepackage[utf8]{inputenc}
\usepackage[T1]{fontenc}

% Joonised
\usepackage{graphicx}

% Tabelid
\usepackage{colortbl}
\definecolor{cellhighlight}{rgb}{0.85, 0.91, 0.97}

% Koodinäited
\usepackage{minted}
\setminted{
    frame=single,
    framesep=10pt,
    fontsize=\footnotesize,
}

% Matemaatilised valemid
\usepackage{amsmath}

% Viited
% Siin saate valida viitamisstiili
\usepackage[style=unitartucs/numeric]{biblatex} % Numbriline viitamine [1], tähestikulises järjekorras viited.
% \usepackage[style=unitartucs/numeric,sorting=none]{biblatex} % Numbriline viitamine [1], viitamise järjekorras viited.
% \usepackage[style=unitartucs/alphabetic]{biblatex} % AMS-stiilis viitamine [ABC], tähestikulises järjekorras viited.
% \usepackage[style=unitartucs/authoryear,uniquename=init]{biblatex} % APA-stiilis viitamine (Koit 2010), tähestikulises järjekorras viited.
\addbibresource{viited.bib} % Selles failis on Teie bibliograafia kirjed

% Meta-andmed
\title{Lõputöö pealkiri}
\author{Autori Nimi}
\date{2024}
\supervisor{Juhendaja nimi\degree{kraad} \and Teise juhendaja nimi\degree{kraad}}
\curriculum{Informaatika õppekava}
\thesis{Bakalaureusetöö (9 EAP)}
\keywords{Kujundus, paigutus, mall}


%% Siit algab dokument

\begin{document}

\maketitle

\linespread{1.45} \selectfont

\newpage
\pdfbookmark[1]{\infoname}{info} % Lisame infolehe PDF-faili järjehoidjatesse.

% Eesti keeles info.
\begin{infoEst}
\begin{abstract}
Siia kirjutage oma lõputöö lühikokkuvõte. See võib lühidalt teemasse sisse juhatada, kuid peaks enamikus katma ära Teie lõputöö sisu. Seda lugedes peaks olema üheselt selge, millest Teie lõputöös lugeda võib. Hea on näiteks kirjutada eesmärgi, meetodi, tulemuste ja järelduste kohta. Bakalaureusetöö korral võiksid nii eesti kui inglise keeles olevad lühikokkuvõtted mahtuda ühele leheküljele. Magistritöö või vajaduse korral võib eesti ja inglise keeles olevad osad kirjutada eraldi lehekülgedele. Mõelge oma tööd kirjeldavad võtmesõnad ning leidke CERCS kood(id).
\end{abstract}

\keywords{}
\cercs{KOOD Koodi nimi}

% Koodid leiate: https://www.etis.ee/Portal/Classifiers/Index/26
% Nt: P170 Arvutiteadus, arvutusmeetodid, süsteemid, juhtimine (automaatjuhtimisteooria)
% Nt: P175 Informaatika, süsteemiteooria

\end{infoEst}


% Inglise keeles info.
\begin{infoEng}
\begin{abstract}
Write here the abstract of your work. The abstract could provide a brief introduction to the topic but should, in the majority, cover Your thesis work. When reading the abstract, it should become clear what can be read about in the rest of Your thesis. It is good to write about the purpose of the thesis, method(s), results, and findings of Your work. For a bachelor’s thesis, both the Estonian and English abstracts should fit into a single page. For a master’s thesis or when necessary, the Estonian and English sections of this page could be on separate pages. Come up with good keywords for Your work and find your CERCs code(s).
\end{abstract}

\keywords{Layout, formatting, template}
\cercs{CODE Code name in English}

\end{infoEng}

\newpage
\section*{Visuaalne kokkuvõte} \label{visaalneKokkuvõte}

\tableofcontents

\input{sections/1-sissejuhatus}
\section{Vormistamine}  \label{vormistamine}
Teie lõputöö dokumendi vormistamine hõlmab väga erinevaid tööriistu, millest Te kõigest ei pruugi teadlik olla. Kuna hea vormistamine on 25\% Teie lõputöö hindest, on oluline vormistamist võtta tõsiselt ja varuda selleks aega. Eriti kui Te ei ole varem nii suuri dokumente nii põhjalikult vormistanud või vastavate tööriistade kasutamist õppinud. Käesolev peatükk kirjeldab lähemalt lõputöö erinevate osade vormistamispõhimõtteid.

\subsection{Tiitel- ja infoleht}
Kõige esimese lehekülge – tiitellehe – vormistamine on siin mallis juba tehtud. Tiitellehel peab olema Teie lõpetatav õppekava (sh õppeasutus ja instituut), lõputöö pealkiri, töö liik ning maht vastavalt õppetasemele, Teie kui autori nimi, Teie töö juhendajate nimed (soovi korral võite lisada ka nende teaduskraadid) ning töö avaldamise asukoht ja aasta. Tiitellehele tulev info tuleb teil sisestada väljadesse failis \path{thesis.tex}. Kindlasti kontrollige kirjutatud informatsioon üle. Näiteks, kui otsustasite lisada juhendajate teaduskraadid, siis kas vastava kraadi lühend on MSc (Master of Science) või MA (Master of Arts). Erinevate õppekavade lõpetajatel on kraadi nimetus erinev.

Tiitellehele järgneb infoleht (fail \path{0-info.tex}), kus on Teie töö lühikokkuvõte, võtmesõnad ning CERCS koodid. Lühikokkuvõte peaks andma ülevaate kõigist Teie töös tehtust. Näiteks kui Teie töö sisupeatükkidest kolm on sellised, mis kirjeldavad põhjalikult erinevaid töö käigus tehtud asju ning vastavaid tulemusi, siis peaks lühikokkuvõttest olema see informatsioon loetav. Keskenduge lühikokkuvõttes sellele, millest Teie tehtust saab lõputöös lugeda.

Võtmesõnad on Teie töös olulisi aspekte nimetavad mõisted, nimetused või nimed. Võtmesõnade väljamõtlemine vajab, et Te mõtleksite oma tööst ülevaatlikult. Millised olulised mõisted tulevad Teile pähe, kui Te oma tööst tervikuna mõtlete? Inspiratsiooniks on hea vaadata teisi varem sarnasel teemal tehtud lõputöid ning nende võtmesõnu. Joonis~\ref{fig:võtmesõnad} esitab varasemalt Tartu Ülikooli arvutiteaduse instituudis aastatel 2015–2021 kaitstud lõputööde populaarsemate võtmesõnade sõnapilve\footnote{\url{https://cglearn.eu/theses/top}}.

\begin{figure}[ht]
    \centering
    \includegraphics[width=\textwidth]{joonised/Joonis1-Võtmesõnad.png}
    \caption{Aastatel 2015-2021 arvutiteaduse instituudis kaitstud lõputöödes kasutatud levinumate võtmesõnade sõnapilv.}
    \label{fig:võtmesõnad}
\end{figure}

Lisaks võtmesõnadele tuleb Teil määrata oma tööle ka Common European Research Classification Scheme (CERCS) kood. Need koodid klassifitseerivad erinevaid teadusvaldkondi. Tihti võib lõputöö puhul olla keeruline klassifitseerida seda konkreetsete CERCS klassifikaatoritega. Sellegipoolest tuleks Teil leida klassifikaator, kuhu alla Teie töö kõige rohkem läheb. Vajadusel võite kasutada töös ka mitut CERCS koodi. Klassifikaatorid leiab Eesti Teadusinfosüsteemi veebilehelt: \url{https://www.etis.ee/Portal/Classifiers/Index/26}.

Pärast infolehte on Teil võimalik lisada oma tööle ka visuaalne kokkuvõte. Selleks on üks joonis, mis kujutab asjalikult Teie töö protsessi ja tulemusi. Joonisele tuleb lisada Tartu Ülikooli arvutiteaduse instituudis kaitsvate lõputööde nõuete ja hindamise dokumendis nõutud informatsioon. Lisades arvutiteaduse instituudi logo, tuleb arvestada logo kaitstud alaga – logol peab igalt poolt olema ühe majakese kõrguse jagu vaba ruumi. Visuaalne kokkuvõte on väga kasulik lisa Teie tööle, sest see annab väga kiiresti edasi lugejale Teie töö olemuse. Samuti saate Te tehtud graafikat kasutada hiljem oma kaitsmisel ja töö populariseerimisel.

\subsection{Struktuur}
Teie lõputöö lugejale on oluline, et töö dokument oleks loogiliselt ja arusaadavalt ülesehitatud. Lugeja juhtimist läbi Teie lõputöö aitavad head pealkirjad, sisukord ning sisu mõistlik peatükkideks jaotamine.

\subsubsection{Pealkirjad}
Nii Teie lõputöö enda kui selle peatükkide pealkirjad on hea mõelda lühikesed ja tabavad. Pikad pealkirjad võivad kanda endas liiga palju mõtteid ning võite oma lugeja kaotada juba pealkirja lugemisel. Lõputöö enda pealkiri võiks mahtuda maksimaalselt kahele reale.

Inglise keeles tuleks pealkirjad kirjutada läbiva suure algustähega. Selles osas on teatud sõnade puhul erinevaid stiile, kuid üldiselt peaksid kõik sõnad peale asesõnade algama suurtähega. Hea on kasutada tööriista Capitalize My Title: \url{https://capitalizemytitle.com/}

Eesti keeles aga kirjutatakse ainult pealkirja kõige esimene täht suurtähena ning ülejäänud sõnad niimoodi nagu tavaliselt.

Pealkirjades tasub proovida vältida kirjavahemärkide kasutamist. Olenevalt töö olemusest võib näiteks mõttekriips olla sobilik. Näiteks, kui Te lõite tarkvaratoote, on töö pealkirjas hea mainida selle nimi ning mõttekriipsu kasutades paari sõnaga selle unikaalne väärtus. Seevastu näiteks komade või küsimärkidega seotud lause pealkirja ei sobiks.

Need põhimõtted kehtivad ka Teie töö peatükkide pealkirjade puhul. Sisupeatükkide pealkirjad algavad numbritega. On Teie valida, kas tahate nummerdamist alustada peatükist “Sissejuhatus”  ja lõpetada peatükiga “Kokkuvõte” või jätate need kaks pealkirja nummerdamata. Siin mallis on “Sissejuhatus” ja “Kokkuvõte” nummerdatud. Peatükile “Kokkuvõte” järgnevad nummerdamata pealkirjad “Viited” ja “Lisad”. Kusjuures erinevate lisade jaoks on mõistlik jälle kasutada nummerdamist. Lisade nummerdamisel saab kasutada näiteks Rooma numbreid (Lisa I, Lisa II) või suurtähti (Lisa A, Lisa B).

Alampeatükkide nummerdus sisaldab ülempeatüki numbrit. Näiteks peatüki 2 alampeatükid on 2.1, 2.2 jne. Kusjuures punkt terve nummerduse lõpus on ainult esimese taseme pealkirjade korral. Alates kolmanda taseme alampealkirjast võib soovi korral nummerduse ära jätta. Siin mallis on pealkirjade stiilid (\verb|\section{...}|, \verb|\subsection{...}| jne) töötavad niiviisi, et nummerdamine oleks sobilik.

\subsubsection{Sisukord}
Teie töö alguses, pärast infolehte ja visuaalset kokkuvõtet ning enne esimese peatüki algust, on sisukord. Sisukord on võimekamades tekstitöötlusprogrammides automaatselt genereeritav. Siin mallis on sisukord genereeritud käsuga \verb|\tableofcontents|. See käsk genereerib sisukorra töö sektsioonide/peatükkide (\verb|\section{...}|, \verb|\subsection{...}| jne) järgi.

\subsection{Tekst}
Lõputöö tekst peab olema rööpjoondatud (teksti äär nii vasakult kui paremalt lehe servast sirge). See võib tekitada olukorra, kus ühes reas jääb sõnade vahele visuaalselt liiga palju ruumi. Sellisel juhul tuleks kasutada poolitamist, et sõnadevahelist ruumi korrastada. Paljud tekstiredaktorid pakuvad ka automaatse poolitamise võimalust, kuid sellega võib juhtuda, et poolitusi tuleb liiga palju. Siin mallis on LaTeX-i automaatne poolitamine minimeeritud ning kasutatud paketti \verb|microtype|, et sõnade vahele jääv ruum ei läheks liiga suureks. Sõnade poolitamine raskendab teksti lugemist ja seda võiks teha minimaalselt. Võiks vältida olukordi, kus mitu rida järjest on poolitatud.

Teksti vormistamisel on Tartu Ülikooli arvutiteaduse instituudis kaitstavate lõputööde nõuete ja hindamise dokumendis esitatud veel mitmeid teksti vormistuse reegleid. Näiteks peaks teksti reavahe olema vahemikus 1,0–1,5. Siin mallis on kasutanud 1,4 reavahe, mis visuaalselt vastab Microsoft Wordi 1,5 reavahele.

On võimalik kasutada kaldkirja (\verb|\emph{...}|), rasvast kirja (\verb|\textbf{...}|) jm teksti vormistamise vahendeid, et oma tekstis mõisteid ja mõtteid rõhutada. Siiski tasub näiteks rasvase kirja kasutamisega olla pigem tagasihoidlik, et Teie töö ei muutuks liiga kirjuks.

Lõike kirjutades võite jälgida, ega Teie lõigu lõppu ei jää viimasele reale ühte üksikut sõna. Sellist sõna nimetatakse orvuks sõnaks ja visuaalselt ei ole see väga kena. Orbude sõnade või ka tekstiridade (üksik tekstirida lehe üleval ääres, mille järel hakkab uus peatükk või tuleb tühi leht) kontroll on midagi, mida on mõistlik teha puhtandi vormistamisel viimasena.

\subsection{Elemendid}
Teie lõputöö väljanägemisele ja arusaamise parandamisele aitavad väga palju kaasa erinevad elemendid nagu joonised, tabelid ja koodinäited.

\subsubsection{Joonised}
Joonisteks nimetame kõiksuguseid lõputöösse lisatud pilte, olgu nendeks graafikud, fotod või ekraanitõmmised. Need kõik tuleb nimetada joonisteks, lisada number ja pealdis. Siin LaTeX-i mallis käib käib see nii, et tuleb pilt kopeerida või laadida üles \verb|joonised| kausta (vt joonis~\ref{fig:kaustad}). Seejärel saab pilti kasutada järgneva koodiga:

\begin{minted}[escapeinside=||]{tex}
\begin{figure}
    \centering
    \includegraphics[width=\textwidth]{joonised/|\textbf{Joonis1-Nimi.png}|}
    \caption{|\textbf{Joonise pealdise tekst.}|}
    \label{fig:|\textbf{jooniseNimi}|}
\end{figure}
\end{minted}

Ülal koodis tuleb kirjutada \verb|joonised| kaustas asuva pildifaili failinimi koos laiendiga, lühike joonise pealdise tekst ja joonisele ristviitamiseks joonise lühinimi ehk \emph{label}.

Pealdise tekstist peaks olema selge, mida on joonisel kujutatud. Pealdis tekib joonise alla ja koos elemendi sildiga „Joonis” ning automaatse joonise numbriga.

\begin{wrapfigure}{r}{0.33\textwidth}
    \centering
    \includegraphics[width=0.33\textwidth]{joonised/Joonis2-joonisedKaust.png}
    \caption{Malli kaustad.}
    \label{fig:kaustad}
\end{wrapfigure}

Te võite jooniseid paigutada oma töös nii lehe keskele kui ka mähkida näiteks teksti kõrvale. Tekstiga mähitud joonise loomiseks tuleb eelnenud joonise koodinäites \verb|figure| asemel kasutada \verb|wrapfigure|. Joonise laiuse võib panna näiteks \verb|width=0.33\textwidth|.

Teksti on soovitav joonise ümber mähkida olukorras, kus joonise laius on umbes 1/3 lehekülje laiusest. Muudel juhtudel näeb joonis tõenäoliselt parem välja lehekülje keskel ilma mähkimiseta.

Jooniseid paigutades tuleb olla väga ettevaatlik, et Te ei skaleeriks joonist ebaühtlaselt. Joonise suurust skaleerides muutes peab joonis nii vertikaalselt kui horisontaalselt skaleeruma samamoodi. Tüüpiliselt LaTeX-is jooniste skaleerimisel seda olukorda ei juhtu, kuid teiste programmidega joonise töötlemisel tuleb seda asjaolu silmas pidada.

Jooniseid tehes on soovitatav luua enda lõputöö jaoks ühtlane värvipalett ja stiil. See tähendab, et kui Te teete näiteks graafikuid, siis nendel oleks kasutatud läbivalt samu värve (muidugi samas tähenduses) ja stiili. Hea tööriist värvipaleti loomiseks on Coolors: \url{https://coolors.co/generate}. Vastavalt olukorrale tuleb muidugi arvestada, et Teie värvipaleti värvid toetaksid ka graafikute lugemist ning korrektset tõlgendamist. Näiteks ei tohiks valida liiga sarnaseid värve graafikul kahe sisuliselt erineva elemendi tähistamiseks. Andmete visualiseerimise põhimõtete kohta on näiteks rääkinud Tamara Munzner oma loengus “Keynote on Visualization Principles”~\cite{tamara_munzner_keynote_2012}. Hea ja ühtlane värvivalik ning stiil ei ole olulised mitte ainult andmete visualiseerimisel, vaid ka näiteks ekraanitõmmistele lisatud graafiliste elementide ja ka muidu jooniste puhul üldiselt.

\subsubsection{Tabelid}
Peale jooniste saab lõputöösse lisada ka andmetabeleid. Tabelite lisamisel tuleks arvestada, et väga suuri tabeleid on raske lugeda ja võib-olla peaks nendel kujutatut esitama hoopis graafikuna. Mõnikord on aga oluline andmed esitada just tabeli kujul. Väiksemad tabelid sobivad väga hästi lõputöö sisusse. Suuremad tabelid saab esitada lõputöö lisades, kas sektsioonis “Lisad” või isegi eraldi failina lõpuööga kaasapandud failide komplektis.

Tabeli puhul kehtivad samad soovitused nagu jooniste puhul. Ainus erinevus on, et tabeli pealdis käib tabeli kohale (jooniste puhul käib alla). Vaata näiteks tabel~\ref{tabel:elementideErinevused}.

\begin{table}[htb!]
    \centering
    \caption{Jooniste ja tabelite vormistamise erinevused lõputöö dokumendis.}
    \label{tabel:elementideErinevused}
    \begin{tblr}{width=1.0\textwidth, hlines, vlines,
                    colspec = { Q[r,font=\bfseries] Q[c] X[c] },
                    row{1} = {font=\bfseries},
                    cell{3}{2} = {bg = colorCellHighlight},
                }
                &   Pealdise asukoht    &   Sisu        \\
    Joonis      & All                   &   Pildid, graafikud, fotod, ekraanitõmmised        \\
    Tabel       & Kohal                 &    Andmed       \\
    Koodinäide  & All või puudub        &    Programmi kood, pseudokood       \\
    \end{tblr}
\end{table}

Ülal olev tabel \ref{tabel:elementideErinevused} on loodud järgneva koodiga:
\begin{minted}{tex}
\begin{table}[htb!]
    \centering
    \caption{Jooniste ja tabelite vormistamise erinevused lõputöös.}
    \label{tabel:elementideErinevused}
    \begin{tblr}{width=1.0\textwidth, hlines, vlines,
                    colspec = { Q[r,font=\bfseries] Q[c] X[c] },
                    row{1} = {font=\bfseries},
                    cell{3}{2} = {bg = colorCellHighlight},
                }
                & Pealdise asukoht    &   Sisu                        \\
    Joonis      & All                 &   Pildid, graafikud...        \\
    Tabel       & Kohal               &   Andmed                      \\
    Koodinäide  & All või puudub      &   Programmi kood, pseudokood  \\
    \end{tblr}
\end{table}
\end{minted}

See kood on sarnane eelnelnud jooniste koodiga, kuid sisaldab kahte keskkonda. Esiteks on keskond \verb|table|, mille sees on ka tabelile eelnev pealdis. Seejärel tuleb keskkond \verb|tblr|, kus on siis juba tabel ise. Antud juhul on määratud tabeli laiuseks terve lehe laius \verb|1.0\textwidth| ning tabelile kõik jooned \verb|hlines| ja \verb|vlines|. Tabeli tulbad on koostatud nii, et kaks esimest on kitsad (\verb|Q|) ning viimasesse pannakse kogu ülejäänud ruum (\verb|X|). Tulbad on joondatud nii, et esimene tulp on paremjoondusega \verb|r| ning teised keskjoondusega \verb|c|. Esimene tulp ja esimene rida on kirjutatud rasvase kirjaga ning tulp indeksiga (3,2) on värvitud helesiniseks.

\subsubsection{Koodinäited}
Programmi kood on soovitatav kirjutada püsisammulise (\emph{monospace}) kirjastiiliga. Tuleks valida üks selline kirjastiil ja kasutada sama läbivalt lõputöö vältel. Näited sellistest püsisammulistest kirjastiilidest on Consolas ja \texttt{Courier New}. Käesolevas mallis on kasutatud \verb|fontspace| paketti, milles on olemas kirjastiil \texttt{Courier New}\footnote{\url{https://www.overleaf.com/learn/latex/Questions/Which_OTF_or_TTF_fonts_are_supported_via_fontspec\%3F}}.

Koodinäite vormistamine ei ole nii reguleeritud nagu piltide ja tabelite puhul. Lühikesi koodinäiteid saab kirjutada teksti sisse. Näiteks võib kirjutada, et muutuja deklareerimine ja väärtuse omistamine käib koodiga \verb|var a = 10;|. Pikema koodinäite puhul tuleks teha eraldi plokk, näiteks nii:

\begin{minted}{javascript}
function add(a, b) {
    return a + b;
}
\end{minted}

Ülalolevas näites on kasutatud \verb|minted| keskkonda \verb|minted| paketist. See keskkond on stiliseeritud lisama koodinäitele ümber joont ja kasutama püsisammulist kirjastiili. Teatud programmeerimiskeeli oskab see pakett ka ise värvida\footnote{\url{https://www.overleaf.com/learn/latex/Code_Highlighting_with_minted\#Reference_guide}}. Koodiplokki ümbritsev tekst peab piisavalt selgelt vastavat koodi sisuliselt käsitlema.

\subsubsection{Matemaatilised valemid}
Matemaatiliste valemite vormistamine LaTeX-is on suhteliselt lihtne. Lühikese valemi kirjutamiseks teksti sisse tuleb see ümbritseda \verb|$|-märkidega. Näiteks liitmist kirjeldab valem $a+b=c$. Suurema ja viidatava valemi jaoks tuleb valem lisada \verb|equation| keskkonna sisse.
\begin{equation}
    \frac{a + b}{c} = d
    \label{eq:abcdValem}
\end{equation}

Ülalolev näide on tehtud koodiga:
\begin{minted}{tex}
\begin{equation}
    \frac{a + b}{c} = d
    \label{eq:abcdValem}
\end{equation}
\end{minted}

Nagu näha, siis valemile saab lisada ka lühinime (\emph{label}), talle tekib automaatselt number ning seda lühinime saab kasutada, et valemit numbri järgi viidata. Näiteks ülalolev näide siin mallis on numbriga~\ref{eq:abcdValem}.

\subsection{Viited}
Õigesti viitamine on Teie töös väga oluline. Ebaõnnestunud viitamise korral võib Teie töö sisaldada akadeemilist petturlust ning selle eest võib Teie töö saada ülevaatamisel negatiivse hinnangu. Käesolevas mallis on eelnevalt kasutatud kolme tüüpi viiteid. Nendeks on töö enda erinevate elementide ja osade vahel olevad ristviited, niinimetatud nõrkade välisviidete jaoks kasutatud joonealused viited ning kasutatud allikate põhiviited ehk tugevad viited.

\subsubsection{Ristviited}
Teie töös olevad joonised, tabelid ja matemaatilised valemid peavad kõik olema tekstis ristviidatud. Selle reegli eesmärk on tagada, et Teie tööd illustreerivad elemendid päriselt ka töö sisu toetaks. Lisatud element peab olema Teie tekstiga seotud ning seega õiges kohas ristviidatud. Ristviidataval elemendil peab olema lühinimi (\emph{label}) ja selle nimega saab luua ristviite käsuga \verb|\ref{lühinimi}|. Viiteks tuleb elemendi järjekorranumber, mis muutub teksti sees ka lingiks (see link säilub ka PDF{\babelhyphen{nobreak}}faili eksportimisel). See tähendab, et kui ristviite peale vajutada, viiakse lugeja viidatud elemendi juurde.

\subsubsection{Joonealused viited}
Väliseid viiteid saab nimetada nõrkadeks ja tugevateks. See kirjeldus ei ütle midagi välise viite mõju või kasulikkuse kohta Teie tööle. Tihtipeale on teatud väliste allikate puhul ka vaja teha Teil subjektiivne otsus, kas see on pigem nn nõrk või tugev viide. Nõrgad viited on näiteks viited Wikipedia artiklitele, Stack Overflow postitustele, toodete kodulehtedele, sõnastikele ja privaatsele suhtlusele, kus viidatud allikas ei ole väga selgelt konkreetse autori poolt mõnes väljaandes publitseeritud. Üldiselt võib mõelda, et kõiksugused veebiviited on just pigem nõrgad viited. Samas veebiviide võib olla ka näiteks tunnustatud autori poolt kirjutatud blogipostitus, mis juhul võib tegu olla pigem tugeva viitega.

Nõrgad viited tuleks vormistada joonealuste viidetena. Selleks tuleb kasutada käsku \verb|\footnote{...}| ja selle sisse kopeerida veebiviite link. Selleks, et link muutuks ka päriselt õigesti vormistatud lingiks, selleks on siin mallis käsk \verb|\url{...}|. Ehk siis lingi lisamine joonealuseks viiteks oleks \verb|\footnote{\url{...}}|. Sellise viite puhul ei ole väga oluline hakata viitesse otsima autorit (nt mõne toote veebilehe, tarkvara dokumentatsiooni või Wikipedia artikli puhul ei pruugigi olla lihtsasti tuvastatavat autorit). Tüüpiliselt piisab lisada joone alla lihtsalt link vastavale veebilehele, kust Te viidatud info saite.

Sellised joonealused viited on väga head, kuid Teie töö peaks põhinema lisaks nendele ka mõistlikul hulgas tugevatel viidetel ehk põhiviidetel.

\subsubsection{Põhiviited}
Siin mallis on nimetatud põhiviideteks neid viiteid, mis lähevad Teie töö lõpus olevasse Viidete sektsiooni. Seda sektsiooni nimetatakse vahel ka „Viidatud kirjandus“ või „Kasutatud allikad“. Sinna lähevad tugevad viited, mis on tüüpiliselt konkreetse autori poolt tunnustatud väljaandena avaldatud. Näiteks teadusartiklid, raamatud, konverentsiettekanded, lõputööd, aga ka võivad olla blogipostitused või uudisartiklid. Vastavalt Teie õppeastmele, millel lõputööd esitate, võiks Teil põhiviiteid olla piisavalt. Täpset arvu on keeruline öelda, sest see sõltub palju Teie töö olemusest, viidatud allikate tugevusest ning kui põhjalikult olete viiteid kasutanud. Samas võib öelda, et näiteks ainult kaks põhiviidet on iga kõrghariduse õppeastme puhul kindlasti liiga vähe.

Põhiviidete vormistamisel on levinud erinevates valdkondades levinud erinevad stiilid. Oma lõputöös on Teil võimalik valida omale meeldiv viitamisstiil. Arvutiteaduses on väga levinud ACM ja IEEE numbrilise viitamise stiilid. Need on sellised, kus allikad nummerdatakse nurksulgudes olevate numbritega (nt „[1]“, „[2]“ jne) ja nendega ka teksti sees viidatakse. Matemaatikas on levinud ka trigraafiline AMS viitamisstiil, kus nurksulgude sees on autorite esitähed või initsiaalid ning osa aastanumbrist (nt „[Mun12]“). Sotisaalteadustes on pigem levinud näiteks APA stiil, kus viidatakse autori nime ja aastaarvuga (nt „(Munzner, 2012)“).

Viidete vormistamiseks on siin mallis kasutatud \verb|biblatex| paketti. See pakett võimaldab Teil lihtsasti lisada oma viited, neile viidata ja muuta viitamisstiili. Käesoleva malli failis \verb|seadistus.tex| saate valida, millist viitamisstiili soovite oma lõputöös kasutada.

Lisaks viitamisele on Teil vajalik saada ka viidatavate allikate bibliograafia kirjed LaTeX-i. Selleks on soovitatav kasutada allikate haldamise tööriista Zotero\footnote{\url{https://www.zotero.org/}}. Zotero võimaldab Teil koguda allikaid ja nendega seotud andmed (sh failid ja Teie märkmed) väga lihtsasti Teie isiklikku andmebaasi Zotero töölauarakenduses. Juba näiteks mustandi kirjutamisel Google Docsis on Teil võimalik Zoterot kasutada, et viited õigetesse kohtadesse lisada ja vormistada.

LaTeX-i vormistama tulles saate Zoterost eksportida vajalikud bibliograafiakirjed. Selleks selekteerige Zoteros oma lõputööga seotud viited ning valige kontekstimenüüst \emph{Export Items...} (vt joonis~\ref{fig:zoteroKontekst}). Avanevast menüüst valige ekspordi formaadiks \verb|BibLaTeX|.

\begin{figure}[ht]
    \centering
    \includegraphics[width=\textwidth]{joonised/Joonis3-ZoteroBibliograafiaEksport.png}
    \caption{Valitud allikate kontekstimenüü Zotero programmis.}
    \label{fig:zoteroKontekst}
\end{figure}

Seejärel Zotero ekspordib teie valitud allikate kirjed \verb|.bib| faili, mille sisu saate tõsta siin keskkonnas faili \verb|viited.bib| ning seejärel nende allikate viitamine juba töötabki. Igale allikale loob Zotero lühinime, millega saate viidata. Näiteks siin mallis Tamara Munzneri loengule viitamiseks on kasutatud käsku \verb|\cite{tamara_munzner_keynote_2012}|.

Viidatud teadusartiklitel on tüüpiliselt Document Object Identifier (DOI) numbrid. Nende põhjal tekivad DOI lingid\footnote{\url{https://dx.doi.org/}} (kujul doi.org/[number]). Teadusartiklitele viidates on soovitatav lisada viitesse ka vastavad DOI lingid ilma vaatamise kuupäevata. Allikate linkide lisamisel tuleb kindlasti tähele panna, et oleks õiged ja töötavad lingid. Võib juhtuda olukord, kus kasutate teadusartiklitele ligipääsuks ülikooli vaheserverit (\emph{proxy}) ja sellisel juhul võivad Teie lingid veebilehitseja aadressiribal olla selle vaheserveri lingid. Kontrollige kindlasti, et allikate lingid oleksid puhtad, töötavad ning võimalusel eelistage DOI-linke.

\subsection{Lisad}
Pärast viidete sektsiooni tuleb Teie lõputöös sektsioon “Lisad”. Selles sektsioonis saab esitada näiteks suuremad tabelid või ekraanitõmmised, mis töö sisu osasse ei mahu. Tihtipeale on seal mõistlik ka esitada Teie tööga seotud sõnastik, kus olete defineerinud oma töös kasutatud spetsiifilised erialaterminid. Olukorras, kus olete lõputöö käigus loonud tarkvara, saab sektsiooni “Lisad” lisada ka tarkvara käivitus- ja kasutusjuhendi. Iga nimetatud lisa kohta saab sektsiooni “Lisad” teha eraldi alajaotuse ja omale meeldival viisil need nummerdada. Näiteks: „Lisa I: Sõnastik“, „Lisa B – Programmi kasutusjuhend“ vms. Juhiseid hea kasutusjuhendi kirjutamiseks on mitmeid, kuid üks soovitus on näiteks Graves ja Gravesi õpik „A Strategic Guide to Technical Communication“ tehnilist laadi teksti kirjutamiseks~\cite{graves_strategic_2012}.

Väga oluline lisa on kaasapandud failide kirjeldus. Lõputöö enda dokumendiga koos on Teil võimalik esitada arhiivifail. Seal arhiivifailis on tüüpiliselt pikemad tekstid (nt kui programmi kasutusjuhend on pikk, on vaja olnud lisada pikemad kasutatud varade litsentsid, uuringus kasutatud küsimustik vms), suuremad andmefailid (nt uuringu käigus mõõdetud analüüsimata tulemused, nende tulemuste analüüsifailid, heli- ja videosalvestised tarkvarast või selle testimisest vms), loodud tarkvara, selle lähtekood, disaini kontseptsioonid ja muud Teie lõputöö käigus tekkinud failid. Lisade sektsioonis kasutage ühte lisadest, et kirjeldada kaasapandud arhiivi sisu.

\subsection{Litsents} \label{subchapter:litsents}
Pärast sektsiooni “Lisad” tuleb Teie lõputöö dokumendi kõige lõppu lisada litsents, mis lubab Tartu Ülikoolil säilitada ja avaldada Teie lõputööd. Selle teksti uuendatakse tihti ja võib sõltuda sellest, kas Teie lõputöö tekst on salastatud või mitte. Uusima litsentsi tekstid leiate aadressilt: \url{https://adr.ut.ee/?page=pub_list_dynobj&desktop=57835&tid=70993&data_only=true&search=Otsi&field_100193_search_type=ANY&field_100193_text_search_value=ppimine}

Ülaoleval lingil on lihtlitsents dokumendis numbriga 30. Enamikel juhtudest peaks see litsents Teile sobima. Olukorras, kus Teie töö sisaldab informatsiooni, mida ei tohi mingi ajaperioodi jooksul või isegi tähtajatult avaldada, peate esitama vastava litsentsi kasutamiseks avalduse õppeprodekaanile. Avaldus on dokumendis number 32 ja kui õppeprodekaan on Teie avalduse rahuldanud, saate kasutada vastavate piirangutega litsentsi dokumendist numbriga 31.

\subsection{Meta-andmed}
Teie lõputöö faili küljes on ka meta-andmed ehk andmed lõputöö faili kohta. Käesolev mall lisab need Teie tööle automaatselt \verb|lõputöö.tex| failis täidetud andmeväljade pealt. PDF{\babelhyphen{nobreak}}faili meta-andmeid saate näha näiteks tarkvaraga Acrobat Reader, kui valite avatud dokuemndi kontekstimenüüst \emph{Document Properties} (vt joonis~\ref{fig:metaAndmed}). On oluline, et Teie töö meta-andmed vastaksid tegelikkusele.

\begin{figure}[ht]
    \centering
    \includegraphics[width=0.8\textwidth]{joonised/Joonis4-MetaAndmed.png}
    \caption{Malli meta-andmed Acrobat Reader programmis.}
    \label{fig:metaAndmed}
\end{figure}

Soovitav on üle vaadata ka ülejäänud failiga seotud meta-andmed, et nendes oleks kõik korras. Varasemalt on näiteks juhtunud, et meta-andmetes olev faili autor ja muutja ei ole olnud lõputöö autor. Selline vastuolu tekitab kahtlust, kes on lõputöö dokumendi tegelikult kirjutanud.

\section{Tekstiloome} \label{tekstiloome}
Selleks hetkeks, mil Te käesolevat malli kasutama hakkate, on Teil tõenäoliselt oma lõputöö sisuline osa valmis mõnes mugavamas ja koostöösõbralikumas keskkonnas, nt Google Docs. Vormistamise juurde tasub minna siis, kui lõputöö sisuline osa enam väga ei muutu. Vastasel juhul võivad muudatused nõuda vormistamise töö uuesti tegemist. Siiski on mõned vormistusega seotud soovitused aktuaalsed ka varem, kui alles mustandi puhtandiks vormistamisel.

\subsection{Peatükid}
Teie lõputöö peatükid võiksid olla üksteisega mahult tasakaalus. Kuna käesolev mall keskendub vormistamisele, siis siinkohal malli sisu selles eksib (eelnev peatükk~\ref{vormistamine} on tunduvalt mahukam kui käesolev peatükk~\ref{tekstiloome}). Teie töös võiksid kõik peatükkide “Sissejuhatus” ja “Kokkuvõtte” vahel olevad peatükid olla mahult enam-vähem võrdse kaaluga. Selline põhimõte aitab Teil ka mitte tegeleda liiga palju ainult oma lõputöö ühe osaga, vaid võrdsemalt kõigi oluliste osadega. See, mis nendes sisupeatükkides täpsemalt on ehk milliste osade vahel peaksite oma tööaega jaotama, sõltub Teie lõputöö tüübist. Tartu Ülikooli arvutiteaduse instituudis kaitstavate lõputööde nõuete ja hindamise dokument defineerib teatud arvu erineva sisu ja fookusega lõputööde tüüpe. Uurimis- või õpperühm, kus Teie oma lõputööd teete, võib defineerida neid veel omakorda. Hoidke pilk peal, et Teie lõputöö dokumendi sisu hõlmaks tasakaalukalt Teie lõputöö tüübis ettenähtavat.

\begin{wrapfigure}{r}{0.33\textwidth}
    \centering
    \includegraphics[width=0.33\textwidth]{joonised/Joonis5-AcrobatReaderMenüü.png}
    \caption{Acrobat Reader parempoolne menüü.}
    \label{fig:acrobatReaderMenüü}
\end{wrapfigure}
Hetkel, mil Teie lõputöö hakkab vormi võtma, oleks hea vaadata lõputöö dokumenti tervikuna. Seda saab teha näiteks Acrobat Reader PDF vaaturis lehekülgede järjekorra muutmise (\emph{Add, delete, or go to specific page using thumbnail images}) tööriistaga (vt joonis~\ref{fig:acrobatReaderMenüü}). Tolle tööriista kuva näitab Teile tervet Teie tööd ülevaatlikult. Justkui oleksite oma töö välja printinud ja kõik lehed eraldi suure laua peale laotanud. Sellest ülevaatest Te näete, kas Teie erinevad töö osad on omavahel tasakaalus ja visuaalselt kooskõlas (vt joonis~\ref{fig:acrobatReaderÜlevaade}).

\begin{figure}[t]
    \centering
    \includegraphics[width=\textwidth]{joonised/Joonis6-AcrobatReaderKuva.png}
    \caption{Lõputöö mall tervikuna vaadatult Acrobat Readeris.}
    \label{fig:acrobatReaderÜlevaade}
\end{figure}
Lõputöö üldisest vaatest näeb ka, kas kõikides vajalikes kohtades on tekst olemas. Näiteks peaks iga peatükk algama peatükki sissejuhatava tekstiga. See tekst peab olema enne kui tuleb alampeatüki pealkiri. Peatükki sissejuhatav tekst kirjeldab, miks käesolev peatükk on Teie töös oluline ning mida võib lugeja oodata alampeatükkidest. Lisaks sellele võiks (alam)peatükkide sisu olla seotud omavahel siduvate lausetega ning suurema peatüki lõpus kirjas ka vastavat peatükki kokkuvõttev lõik. Peatükk ei tohiks alata ega lõppeda ühegi elemendiga (joonis, tabel, koodinäide, valem) ega loeteluga. Oluline on, et Teie töö oleks sujuv lugeda.

\subsection{Õigekirjakontroll}
Puhtandi vormistamisel aitab Teid väga palju Overleafi õigekirjakontroll, mis tasub kindlasti üldmenüüst sisse lülitada ja õige keel valida. Olles mustandiga kaua aega tööd teinud, olete võib-olla juba ära harjunud ühte ja sama teksti nägema ning seetõttu võib olla raske ise kirjavigu märgata. Overleafis on olemas nii eesti kui inglise keele jaoks õigekirjakontrolli tööriist. Kirjavigadega töö korral on Teil oht saada madalam hinnang.

Muidugi võite kasutada ka teisi õigekirja parandavaid tööriistu nt Grammarly, kui Teil on neile ligipääs ja need Teie tööd efektiivsemaks teevad.

\subsection{Generatiivse tehisaru kasutamine}
Generatiivse tehisaru tööriistad ja nende kasutamine võimaldab ka Teie tööd efektiivsemaks muuta. Siinkohal on keskendutud tehisaru kasutamisele tekstitoimetamise seisukohast. Olukorras, kus olete tehisaru kasutanud oma töö sisulistel eesmärkidel (nt uuringu küsimuste koostamisel), tuleks Teil kindlasti see kasutusmetoodika ära kirjeldada oma töö sisu osas. Seevastu teksti toimetamine ei ole Teie töö sisuline osa ja selle tarvis tehisaru tööriista kasutamist tuleks Teil lihtsalt mainida näiteks töö peatüki “Sissejuhatus” lõpus.

Hea viis tehisarul põhinevat juturobotit kasutada on näiteks anda talle ette oma lõputööst lõik ja viip, mis käsib muuta antud akadeemilise teksti lõigu paremini loetavamaks. Selle juures tuleb tulemus ise üle lugeda ja vajadusel veel korrigeerida. Tehisarul põhinev juturobot võib olla väärastanud Teie lõigus olnud mõtted ning need tuleb Teil taastada. Samuti ei pruugi Te isiklikult olla nõus konkreetse tulemuse stiiliga ning tahate seda ise veel omakorda korrigeerida. Tehisarul põhinevad tööriistad võivad aidata Teil teksti loetavust väga efektiivselt parandada, kuid Teil on vaja olla hoolas, et tehisaru poolt pakutud teksti poolt kantud mõtted on endiselt Teie ja pakutav akadeemilise stiili valik Teile meelepärane.

Kindlasti ei tasu ainult tehisaru poolt mõeldud lõike või sisu vahetult oma lõputöös kasutada. Lõputöö autor olete ikkagi Teie ning see tähendab, et Teie vastutate lõputöös kirjutatu eest. Tüüpiliselt on tehisaru kirjutatud tekst liiga üldistav, ülemäära illustratiivne ning sisaldab faktivigu, mida asjatundlik inimesest teksti autor ei teeks. Te ei taha olukorda, kus peate hakkama tekstis leiduvate probleemide pärast hakkama süüdistama kasutatud tehisaru tööriista.

\input{sections/4-kokkuvõte}

\clearpage
\printbibliography[heading=bibintoc]

\input{sections/10-lisad}
\input{sections/11-litsents}

\end{document}



