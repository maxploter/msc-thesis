\newpage
\pdfbookmark[1]{\infoname}{info} % Adding the info page among the PDF bookmarks


% English information
\begin{info}
\begin{abstract}
Write here the abstract of your work. The abstract could provide a brief introduction to the topic but should, in the majority, cover Your thesis work. When reading the abstract, it should become clear what can be read about in the rest of Your thesis. Write about the purpose of the thesis, method(s), results, and findings of Your work. For a bachelor’s thesis, both the Estonian and English abstracts should fit into a single page. For a master’s thesis or when necessary, the Estonian and English sections of this page could be on separate pages. Come up with good keywords for Your work and find your CERCS code(s).
\end{abstract}

\keywords{}

\cercs{CODE Code name}
% Codes can be found at: https://www.etis.ee/Portal/Classifiers/Index/26
% Example: P170 Computer science, numerical analysis, systems, control
% Example: P175 Informatics, systems theory
\end{info}



% Estonian information
\begin{otherInfo}{estonian}{Lõputöö pealkiri eesti keeles}
\begin{abstract}
Siia kirjutage oma lõputöö lühikokkuvõte. See võib lühidalt teemasse sisse juhatada, kuid peaks enamikus katma ära Teie lõputöö sisu. Seda lugedes peaks olema üheselt selge, millest Teie lõputöös lugeda võib. Kirjutage eesmärgi, meetodi, tulemuste ja järelduste kohta. Bakalaureusetöö korral võiksid nii eesti kui inglise keeles olevad lühikokkuvõtted mahtuda ühele leheküljele. Magistritöö või vajaduse korral võib eesti ja inglise keeles olevad osad kirjutada eraldi lehekülgedele. Mõelge oma tööd kirjeldavad võtmesõnad ning leidke CERCS kood(id).
\end{abstract}

\keywords{Kujundus, paigutus, mall}

\cercs{KOOD Koodi nimi eesti keeles}
\end{otherInfo}

