\section{Introduction} \label{Introduction}

Your thesis consists of a title page, info page, visual abstract (when needed or desired), table of contents, chapters, references, and appendices. This thesis template is a basis and provides guidelines for formatting all these components and the thesis as a whole. As formatting makes up of 25\% of Your thesis assessment, You should allocate sufficient time for it.

This document includes guidelines for formatting Your thesis with LaTeX. The use of this technology is not mandatory, but it does provide You with tools for crrect formatting. In accordance with Your own wishes, You can use other text editors such as Apache OpenOffice Writer or Pages. You can also create and format your thesis in Microsoft Word. There is a separate template for that. When making Your thesis draft with Your supervisor, You might have used Google Docs software. While Google Docs is an excellent tool for collaborative work, it does not have the tools for correctly formatting Your thesis. Thus, when You have reached a point in Your thesis writing when it is time to format a fair copy from Your draft, You should use sufficiently capable software packages.

The template provides instructions and guidelines related to the different components of Your work. The contents of this template are meant as recommendations that will help You format Your thesis. Specific rules, also those regarding the formatting, are written in the \textbf{Guidelines for preparing and grading of graduation theses at the Institute of Computer Science of the University of Tartu} document. The latest version of that is available on the webpage of the Institute of Computer Science: \url{https://cs.ut.ee/et/sisu/loputoode-tahtajad-ja-juhendid}


In addition to formatting, this template also provides recommendations regarding the contents of different sections of Your thesis. Regarding the thesis contents, You should also refer to the institute’s official guidelines document. Hopefully the recommendations here also help You write Your thesis better.

The Introduction chapter should open Your thesis topic and explain its relevance (implicitly tell the reader why they should continue to read Your work). After opening the topic, You should explain to the reader what Your thesis offers to the world, what issues it is solving, or what questions it is answering. From the Introduction, it should be obvious what goals You had to make Your thesis. Then, either at the end of the Introduction or weaved into the chapter itself, there should be a brief overview of all the content chapters. The corresponding content chapters must be cross-referenced when described.

The length of the Introduction chapter depends on the complexity of the work. Typically, for a bachelor’s thesis, a single page is sufficient to describe to the reader the importance of the topic, the goal/contribution of the work, and the contents of the thesis. For a master’s thesis, several pages are typically needed, as the topics are more complex and need lengthier explanations. Technically, You can add subchapters to the Introduction, but that is generally not very useful. Ideally, Your Introduction chapter should be the same length as the Conclusion chapter, and they should go together with each other (reading sequence of the Introduction and then the Conclusion, the reader should already get a very good overview of Your work). It is understandable that You wish to explain Your field to the reader more thoroughly – to give the reader all the prerequisites to understand Your contribution. For that, it is good to use the subsequent main chapter right after the Introduction. When it comes to the introduction chapter itself, try to keep it below a few pages in length, introductory, and on-point.

In chapter ~\ref{formatting} this thesis template describes aspects of formatting the thesis and offers recommendations. It covers different components of the thesis, from the entire thesis structure to the thesis elements and references. For each component, the template presents important things to consider when formatting that component. Chapter ~\ref{textCreation} subsequently provides guidelines related to text creation, such as the use of generative artificial intelligence.

The template is structured similarly to a thesis. When using this template to format Your fair copy, delete all the current content and replace it with the content You have created in Your draft document. The replaced content should be virtually finished content. If it is subject to changes, the formatting work could need to be redone.