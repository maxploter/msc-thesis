\section{Conclusion} \label{Conclusion}
This thesis template has given an overview and recommendations on various formatting and text creation techniques. For specific rules, it is useful to see the Guidelines for preparing and grading of graduation theses at the Institute of Computer Science of the University of Tartu document. Although this template focused on the LaTeX technology, the formatting rules and recommendations also apply to other capable text editors (for example, Microsoft Word, OpenOffice Writer, Pages, etc.). When creating a fair copy from Your draft, ensure that Your chosen text editor includes all the necessary tools. For example, Google Docs is not suitable for that work as it is missing several required tools presented in this template.

When formatting Your thesis, look at Your thesis document both as a whole (structure, balance, visual style) as well as individual components separately (info page, content elements, references, metadata). The different parts of Your work need different approaches. For example, to format the main references effectively, an external tool like Zotero might be useful. Formatting means not only positioning and cross-referencing an element but also the formatting of the element itself. For example, the plots in Your thesis must be well-designed and neatly formatted, just like the thesis itself. It is crucial to allocate enough time to study the tools and techniques to format Your thesis thoroughly.

Hopefully, this template helps You. To use this template, make a copy of it and delete or substitute all the content text. The pre-configured text styles should help You format Your draft to a fair copy. Certainly, look over Your document’s metadata and ensure they correspond to Your real thesis contents. Your supervisor can also be very helpful when it comes to thesis formatting. So, be active in asking them for help. Good luck with this last graded 25\% You do before submitting Your Thesis!