\section{Background}  \label{Background}



Video object detection is a computer vision task which involves detecting objects in a video in comparison conventional object detection task.

Advancement in deep learning field pushed offered a boost for video object detection performance \cite{TODO}.
The easiest approach researchers to treat each set of video frames as independent images and use a state of the art object detection model \cite{TODO}. However video posses more temporal information and spacial information than images.

In order to utilise temporal information researchers proposed an idea to use prostprocessing to improve detection results \cite{TODO}. The approach is quite simple they take high confidence detections in adjecent frames and use them to improve results. The main difference between those methods are the mapping strategy. Howerver, postprocessing apraoch for videso detection has a few limitations like error from detection model is propogated to the postprocessing which has no way to recover the error \cite{}, not temporal consistency \cite{} and finally temporal information was not fully utilised.

% not clear what does it mean motion information?
Next researches suggested to introduce motion and time information during training to form an end-to-end solution. Additional models can be divided by methods they use sucha as optical flow, context, or trajectory.a
Optical flow rely on key frames which supplement the current frame. Optical flow is an old task proposed in 1981 \cite{TODO}. However, optical flow has dow.

Context method rely on recurrent models to capture object associations for a longer timeframe than optical flow.


Video object detection algorithms can be classified into four categories \ref{FIGURE} \cite{TODO}.

First, the easiers approach is to postporcess image object detection results.

Second, use axilary network to extract motion information.a

