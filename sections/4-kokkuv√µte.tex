\section{Kokkuvõte} \label{kokkuvõte}
Käesolev mall on andnud ülevaate ja soovitused erinevate lõputööga seotud vormistamise ja tekstiloome tehnikate kohta. Kindlasti tasub konkreetsete reeglite jaoks vaadata Tartu Ülikooli arvutiteaduse instituudis kaitstavate lõputööde nõuete ja hindamise dokumenti. Kuigi käesolev mall keskendus Overleaf LaTeX tehnoloogiale, siis järgides vormistusnõuded ning soovitusi, on võimalik Teil oma lõputööd vormistada ka teistes võimekates tekstiredaktorites (nt Microsoft Word, OpenOffice Writer, Pages jm). Mustandi pealt puhtandi vormistamise tööriista valiku tegemisel olge veendunud, et vastavas tööriistas on Teile vajalikud tööriistad olemas. Näiteks Google Docs selleks tööks ei sobi, sest seal on mitmed siin mallis kirjeldatud ja vormistamiseks vajalikud vahendid puudu.

Oma lõputöö vormistamisel tuleb Teil jälgida nii tööd tervikuna (struktuur, tasakaal, visuaalne stiil) kui ka mitmesuguseid individuaalseid komponente (infoleht, illustratiivsed elemendid, viited, lisad, meta-andmed). Erinevad osad tööst vajavad vormistamiseks erinevat lähenemist. Näiteks on põhiviidete efektiivseks haldamiseks hea kasutada välist tööriista nagu Zotero. Vormistamise alla kuulub ka mitte ainult töö elementide paigutamine ja ristviitamine siia dokumenti, vaid ka elementide enda vormistamine. Näiteks Teie lõputöös kasutatud graafikud peavad olema samuti hästi kujundatud ning vormistatud nagu lõputöö ise. On oluline varuda aega, et õppida tööriistu ja tehnikaid ning vormistada oma lõputöö erinevatest külgedest.

Loodame, et käesolev mall on Teile abiks. Selle malli kasutamiseks soovitame teha mallist koopia ja kustutada või asendada ära kõik sisuline tekst. Malli seadistused, loodud käsud ning juhendid võiksid olla teile abiks LaTeX-is oma töö vormistamisel. Kindlasti vaadake üle ka oma faili meta-andmed, et need oleksid õiges vastavuses Teie lõputöö sisuga. Ka vormistamisega seotud probleemide korral saab Teie lõputöö juhendaja Teile abiks olla, seega olge julged talt abi küsima. Edu selle viimase 25\% hinnatava osaga enne lõputöö esitamist!
