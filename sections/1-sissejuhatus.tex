\section{Sissejuhatus} \label{sissejuhatus}

Teie lõputöö koosneb tiitellehest, infolehest, visuaalsest kokkuvõttest (vajadusel või soovi korral), sisukorrast, peatükkidest, viidetest ja lisadest. Käesolev mall on aluseks ja annab juhised, kuidas kõiki neid komponente ja lõputööd tervikuna vormistada. Vormistamine on 25\% Teie lõputöös hinnatavast, seega soovitame varuda selleks piisavalt määral aega.

Siin dokumendis on juhised Overleaf LaTeX keskkonnas vormistamiseks. Selle tarkvara kasutamine ei ole kohustuslik, kuid annab Teile tööriistad korrektselt vormistada. Vastavalt oma soovile võite kasutada ka teisi võimekaid tekstiredaktoreid nagu Microsoft Word, Apache OpenOffice Writer või Pages. Microsoft Wordis vormistamiseks pakume eraldi malli. Juhendajaga koos lõputöö mustandiga tööd tehes olete võib-olla kasutanud Google Docs tarkvara. Kuigi Google Docs on väga hea tööriist mustandiga koostöösõbralikult töötamiseks, ei ole seal paraku piisavalt tööriistu lõputöö korrektseks vormistamiseks. Seega, kui olete jõudnud lõputöö kirjutamisega nii kaugele, et mustandi pealt puhtand teha, tuleks puhtandi vormistamiseks kasutada piisavalt võimekaid tööriistapakette.

Käesolev mall kirjeldab vormistamisega seotud juhiseid Teie töö erinevates komponentides. Siin mallis kirjeldatud juhiseid on mõeldud soovitustena, mis aitavad Teil vormistada oma lõputöö. Konkreetsed, ka vormistusega seotud, reeglid on kirjeldatud Tartu Ülikooli arvutiteaduse instituudis kaitsvate lõputööde nõuete ja hindamise dokumendis, mille uusima versiooni leiate arvutiteaduse instituudi vastavalt lehelt: \link{https://cs.ut.ee/et/sisu/loputoode-tahtajad-ja-juhendid}.
Lisaks vormistuse soovitustele pakub käesolev mall ka mõningaid soovitusi, mida erinevates töö osades kirjutada. Ka lõputöö sisu osas tasub ennekõike pöörduda instituudi nõuete ja hindamise dokumendi poole, kuid loodame, et esitatud soovitused aitavad Teid tööd paremini kirjutada.

Sissejuhatuse peatükk peaks avama Teie lõputöö teema ning põhjendama selle olulisust (vaikimisi vastama, miks peaks lugeja Teie tööd edasi lugema). Teema avamise järel tuleks lugejale selgitada, mida Teie töö maailmale pakub, mis probleemi lahendab või küsimustele vastuseid annab. Sissejuhatust lugedes peaks olema selge, mis eesmärgiga olete oma lõputöö teinud. Seejärel, kas sissejuhatuse lõpus või põimituna ülejäänud teksti, tuleks anda ülevaate lõputöö sisupeatükkide ning lisade kohta. Vastavaid peatükke ja lisasid tuleks ka kirjeldamisel ristviidata.

Sissejuhatuse pikkus sõltub teema keerulisusest. Tüüpiliselt piisab bakalaureusetöö korral ühest leheküljest, et teema olulisust, lõputöö eesmärki/panust ja sisu lugejale piisavalt selgitada. Magistritöö puhul on tüüpiliselt vaja paari lehekülge, kuna teemad on komplekssemad ja vajavad pikemat selgitust. Sissejuhatusse võib tehniliselt lisada alampeatükke, kuid pigem ei ole see kasulik. Ideaalis võiks sissejuhatuse peatükk olla sama pikk kui kokkuvõtte peatükk ning nad võiksid omavahel kokku minna (lugedes järjest sissejuhatust ja kokkuvõtet saab väga hea ülevaate tervest tööst). On arusaadav, et Te soovite oma valdkonda natuke põhjalikumalt avada – anda lugejale eelteadmised, et ta oleks ettevalmistatud Teie panusest aru saamiseks. Selleks on hea aga juba järgmine suurem peatükk pärast sissejuhatust. Sissejuhatus ise proovige hoida paari lehekülje pikkune, sissejuhatav ja tabav.

Käesolev mall kirjeldab peatükis~\ref{vormistamine} lõputöö dokumendi vormistamisega seotut ning esitab soovitusi. Vaatame läbi erinevad lõputöö komponendid alates struktuurist endast kuni töö elementide ja viideteni. Iga komponendi juures esitame olulised kohad, mida soovitame selle vormistamise juures silmas pidada. Peatükk~\ref{tekstiloome} annab seejärel mõned tekstiloomega seotud juhtnöörid, näiteks tehisaru kasutamise kohta tekstiloomel.

Mall on struktureeritud sarnaselt nagu lõputöögi. Selle malli kasutamiseks lõputöö puhtandi vormistamisel soovitame kustutada mallis esitatud sisu ning asendada see valmis sisuga enda mustandist. Asendatud sisu võiks olla põhimõtteliselt lõplik sisu, sest kui pärast vormistamist see sisu suuresti muutub, võib olla vaja vormistamise töö uuesti teha.
